%%%% ijcai19.tex

\typeout{IJCAI-19 Instructions for Authors}

% These are the instructions for authors for IJCAI-19.

\documentclass{article}
\pdfpagewidth=8.5in
\pdfpageheight=11in
% The file ijcai19.sty is NOT the same than previous years'
\usepackage{rapport}

% Use the postscript times font!
\usepackage{times}
\usepackage{soul}
\usepackage{url}
\usepackage[hidelinks]{hyperref}
\usepackage[utf8]{inputenc}
\usepackage[small]{caption}
\usepackage{graphicx}
\usepackage{amsmath}
\usepackage{booktabs}
\usepackage{algorithm}
\usepackage{algorithmic}
\urlstyle{same}

% the following package is optional:
%\usepackage{latexsym} 

% Following comment is from ijcai97-submit.tex:
% The preparation of these files was supported by Schlumberger Palo Alto
% Research, AT\&T Bell Laboratories, and Morgan Kaufmann Publishers.
% Shirley Jowell, of Morgan Kaufmann Publishers, and Peter F.
% Patel-Schneider, of AT\&T Bell Laboratories collaborated on their
% preparation.

% These instructions can be modified and used in other conferences as long
% as credit to the authors and supporting agencies is retained, this notice
% is not changed, and further modification or reuse is not restricted.
% Neither Shirley Jowell nor Peter F. Patel-Schneider can be listed as
% contacts for providing assistance without their prior permission.

% To use for other conferences, change references to files and the
% conference appropriate and use other authors, contacts, publishers, and
% organizations.
% Also change the deadline and address for returning papers and the length and
% page charge instructions.
% Put where the files are available in the appropriate places.

\title{Titre}

% Single author syntax
\iffalse
\author{
    Sarit Kraus
    \affiliations
    Department of Computer Science, Bar-Ilan University, Israel \emails
    pcchair@ijcai19.org
}
\fi

% Multiple author syntax (remove the single-author syntax above and the \iffalse ... \fi here)
% Check the ijcai19-multiauthor.tex file for detailed instructions
\author{
Antoine Gaulin\and
Othman Mounir\and
Arthur Pulvéricl\And
Anis Redjdal\\
\affiliations
Polytechnique Montréal\\
}

\begin{document}

\maketitle

\begin{abstract}
Décrire la thèse et le résultat en moins de 200 mots. Le document est de 6 pages maximum.
\end{abstract}

\section{Introduction}

Un petit résumé des travaux antérieurs qui existent dans la littérature.

\subsection{Exemples de citation}

La gestion des références est dans le .bib ~\cite{gottlob:nonmon}

\section{Approche théorétique}

Un résumé de l'approche théorétique formant la base du sujet du projet.

\subsection{Exemple de formule}

\begin{align}
    x =& \prod_{i=1}^n \sum_{j=1}^n j_i + \prod_{i=1}^n \sum_{j=1}^n i_j + \prod_{i=1}^n \sum_{j=1}^n j_i + \prod_{i=1}^n \sum_{j=1}^n i_j + \nonumber\\
    + & \prod_{i=1}^n \sum_{j=1}^n j_i
\end{align}

\section{Discussion}

Discussion de nos expériences. Incluant des figures, des tableaux de nos résultats.

Le code source est disponible à cette adresse : \url{https://github.com/Anteige/INF8225}.

\subsection{Exemples de tableau}

\begin{table}
\centering
\begin{tabular}{lrr}  
\toprule
Scenario  & $\delta$ (s) & Runtime (ms) \\
\midrule
Paris       & 0.1  & 13.65      \\
            & 0.2  & 0.01       \\
New York    & 0.1  & 92.50      \\
Singapore   & 0.1  & 33.33      \\
            & 0.2  & 23.01      \\
\bottomrule
\end{tabular}
\caption{Booktabs table}
\label{tab:booktabs}
\end{table}

\subsection{Exemple d'algorithme}

\begin{algorithm}[tb]
\caption{Example algorithm}
\label{alg:algorithm}
\textbf{Input}: Your algorithm's input\\
\textbf{Parameter}: Optional list of parameters\\
\textbf{Output}: Your algorithm's output
\begin{algorithmic}[1] %[1] enables line numbers
\STATE Let $t=0$.
\WHILE{condition}
\STATE Do some action.
\IF {conditional}
\STATE Perform task A.
\ELSE
\STATE Perform task B.
\ENDIF
\ENDWHILE
\STATE \textbf{return} solution
\end{algorithmic}
\end{algorithm}

\section{Analyse}

Une analyze critique de l'approche que vosu avez utilisée pour apprendre le sujet que vous avez sélectionné.


%% The file named.bst is a bibliography style file for BibTeX 0.99c
\bibliographystyle{named}
\bibliography{rapport}

\end{document}

